\documentclass[a4paper]{article}
\usepackage{a4wide}
\usepackage[english]{babel}
\usepackage{amsmath,amssymb}
\usepackage{tikz}
\usetikzlibrary{arrows,trees,automata}
\usepackage{aip}%             't beste als aip de laatste in de lijst is.

\usepackage{graphics}
\usepackage{graphicx}
\usepackage{pgf}
%\usepackage{palatino}
\usepackage{amsmath,amssymb}
\usepackage{bm}
\usepackage[english]{babel}
\usepackage{fancybox}
\usepackage{hyperref}

% colors
\def\red#1{{\color{red}#1}}
\def\blue#1{{\color{blue}#1}}
\definecolor{mygreen}{rgb}{0,.5,0}
\def\green#1{{\color{mygreen}#1}}

 % input, general variable
\newcommand{\xvar}{{X}}          % input variable X
\newcommand{\x}{{\mathit{x}}}    % scalar input x
\newcommand{\xv}{{\mathbf{x}}}   % (D x 1) vector input (\x_1, ...,\x_D)^T
\newcommand{\xmat}{{\mathbf{X}}} % (N x D) input matrix 
\newcommand{\xvec}{%             % (N x 1) input vector (N observations) 
	{\ensuremath{\boldsymbol{\mathsf{x}}}}}  
 
% target data
\newcommand{\yvar}{T}          %  target variable 
\newcommand{\y}{{\mathit{t}}}  %  scalar target value 
\newcommand{\yv}{{\mathbf{y}}} %  vector target
\newcommand{\yvec}{%           %  N x 1 target vector
	{\ensuremath{\boldsymbol{\mathsf{t}}}}} 

% latent variables
\newcommand{\zvar}{Z}            % latent variable 
\newcommand{\z}{{\mathit{z}}}    % latent var value 
\newcommand{\zv}{{\mathbf{z}}}   % K x 1 latent variable
\newcommand{\zmat}{{\mathbf{Z}}} % NxK latent var matrix 

% data sets
\newcommand{\data}{\mathcal{D}} % observed data set
\newcommand{\xset}{\{\xv_1,\dotsc,\xv_N\}} % input sequence
\newcommand{\yset}{\{\y_1,\dotsc,\y_N\}}   % target sequence
\newcommand{\xyset}{\{(\xv_1,\y_1),\dotsc,(\xv_N,\y_N)\}} % paired seq

% sequences (ordered sets)
\newcommand{\xseq}[2]{{\xv_{#1},\dotsc,\xv_{#2}}} 
\newcommand{\zseq}[2]{{\zv_{#1},\dotsc,\zv_{#2}}}

% other signals

\newcommand{\class}{{\mathcal{C}}} %  classes
\newcommand{\noise}{{\epsilon}} %  noise	
\newcommand{\noisevec}{%
	{\ensuremath{\boldsymbol{\epsilon}}}} 


%% parameters

\newcommand{\thpar}{{\theta}} % generic parameters
\newcommand{\thvec}{%
	{\ensuremath{\boldsymbol{\theta}}}}
\newcommand{\thold}{{\thvec^{\text{\red{old}}}}}
\newcommand{\thnew}{{\thvec^{\text{\red{new}}}}} 
\newcommand{\thlib}{{\Theta}}

\newcommand{\w}{{\mathit{w}}} % alternative parameter 
\newcommand{\wv}{{\mathbf{w}}} 
\newcommand{\wmat}{{\mathbf{W}}}
\newcommand{\wlib}{{\mathcal{W}}}

% prior
\newcommand{\prio}{{\pi}}
\newcommand{\priovec}{\ensuremath{\boldsymbol{\pi}}}

 % Gaussian pars

\newcommand{\Normal}{\mathcal{N}}
\newcommand{\Bern}{\mathrm{Bern}} % bernouilli

\newcommand{\mupar}{{\ensuremath{\mu}}}
\newcommand{\muvec}{{\ensuremath{\boldsymbol{\mu}}}}

\newcommand{\sgm}{{\sigma}}
\newcommand{\sgmsq}{{\sigma^2}}
\newcommand{\Sgm}{{\ensuremath{\boldsymbol{\Sigma}}}}
\newcommand{\Prec}{{\ensuremath{\boldsymbol{\Lambda}}}}


% prob, expectation, variance
\newcommand{\p}{p} % probability (mass and density)
\newcommand{\Exp}{\mathbb{E}} % expectation
\newcommand{\cov}{\mathrm{cov}} % covariance
\newcommand{\var}{\mathrm{var}} % variance

% chapter 12 on PCA

\newcommand{\vv}{{\mathbf{v}}}   % 
\newcommand{\vmat}{{\mathbf{V}}}   % 
\newcommand{\mv}{{\mathbf{m}}}   % 
\newcommand{\rmat}{{\mathbf{R}}}

% chapter 13 on PCA

\newcommand{\amat}{{\mathbf{A}}}   % 
\newcommand{\cmat}{{\mathbf{C}}} 
\newcommand{\Gambf}{{\ensuremath{\boldsymbol{\Gamma}}}}

% extra math
\newcommand{\realnumbers}{\mathbb{R}}
\newcommand{\beq}{\begin{equation}}
\newcommand{\eeq}{\end{equation}}

\newcommand{\trace}{\mathrm{Tr}}
\newcommand{\diag}{{\mathrm{diag}}}
\newcommand{\zerovec}{{\mathbf{0}}} % 0 vector
\newcommand{\llh}{{\mathit{L}}} % log likelihood
\newcommand{\resp}{\gamma} % responsibility
\newcommand{\Q}{\mathcal{Q}}  % expected complete llh
\newcommand{\KL}{\mathrm{KL}} % Kullback-Leibler divergence
\newcommand{\FreeEnergy}{\mathcal{L}}
\def\d#1{{\,\mathrm{d}#1}} % differential d in integrals
%\newcommand{\d}[1]{{\,\mathrm{d}#1}}





%%%
%%%
%%%

\newcommand{\A}{\mathcal{A}}
\newcommand{\B}{\mathcal{B}}
\newcommand{\C}{\mathcal{C}}
\newcommand{\D}{\mathcal{D}}
\newcommand{\E}{\mathcal{E}}
\newcommand{\I}{\mathcal{I}}
\newcommand{\M}{\mathcal{M}}
\newcommand{\N}{\mathcal{N}}
\newcommand{\X}{\mathcal{X}}
\newcommand{\Y}{\mathcal{Y}}
\newcommand{\mc}[1]{\mathcal{#1}}


% full page graph inclusion 
\newcommand{\incgraph}[1]{\includegraphics[keepaspectratio,width=\textwidth, height=.8\textheight]{#1}}





%colors
%\def\r#1{{\color{red}#1}}
%\def\b#1{{\color{blue}#1}}
%\definecolor{mygreen}{rgb}{0,.5,0}
%\def\g#1{{\color{mygreen}#1}}

   

%\newcommand{\tjboxed}[1]{\par\begin{center}\tikz \node[draw,text width=13cm,inner sep=3pt,line width=1pt] {\parbox{13cm}{#1}};\end{center}}
\newcommand{\tjboxed}[1]{}



%                        HEADER INFORMATIE
\examendatum{24 June 2010}                    % Datum van het examen.
\examentijd{09u00--12u00}                       % Tijd van het examen.

\begin{document}

\begin{exam}
%
% algemene vorm van invoer:
% gebruik het 'environment'
%       \begin{vraag}{xxx}
%         ...
%       \end{vraag}
% Hierin wordt in xxx beschreven hoe de verschillende onderdelen
% scoren. Per vraag zijn 10 punten te verdelen. Bij het nakijken per
% onderdeel een geheel aantal punten tussen 0 en aangegeven aantal
% toekennen.
%
% Deelvragen worden aangegeven door het environment
% \begin{deelvraag}
%   ...
% \end{deelvraag}
%

% Hier is vraag 1
\begin{vraag}{each sub-question 2 points. Total 10 points}
For each of the following sub-questions, you are asked to provide a \emph{short but essential} answer. You should not need more than five sentences per answer.

\begin{deelvraag}
    Explain shortly how Bayes rule relates to machine learning. In your answer, you may assume a model $\mathcal{M}$ with prior distribution $p(\mathcal{M})$ and an observed data set $D$.
\tjboxed{
$$ \underbrace{p(\mathcal{M}|D)}_{\text{posterior}} = \frac{p(D|\mathcal{M})}{p(D)}\underbrace{p(\mathcal{M})}_{\text{prior}}
$$
Bayes rule relates what we know about a model before (prior) and after (posterior) having seen the data. The difference between the prior and posterior distributions for the model can be interpreted as a `machine learning' effect. (Alternative answers are also possible).
}%end tjboxed
\end{deelvraag}

\begin{deelvraag}
    Explain the relation between Bayesian estimation, Maximum a Posteriori (MAP) estimation and Maximum Likelihood (ML) estimation. You may assume a context of a given model structure with unknown parameters $\theta$ and an observed data set $D$.
\tjboxed{
\begin{eqnarray*}
\hat \theta_{\text{bayes}} = \int_\theta \theta p(\theta|D) \d{\theta} & \text{(Bayes est.)}\\
\hat \theta_{\text{map}} = \arg\max_\theta p(\theta|D) = \arg\max_\theta p(D|\theta)p(\theta) & \text{(MAP)}\\
\hat \theta_{\text{ml}} = \arg\max_\theta p(D|\theta) & \text{(ML)}
\end{eqnarray*}
Bayes estimation picks the mean from the posterior $p(\theta|D)$. MAP picks the mode from $p(\theta|D)$. ML is MAP with uniform prior. (Alternative answers are also possible).
}%end tjboxed
\end{deelvraag}

  The following two sub-questions relate to a (Factor Analysis) model $x_n=\Lambda z_n + v_n$ for an observed data set $D=\{x_1,\dots,x_N\}$. The modeling assumptions include $z_n \sim \mathcal{N}(0,I)$, $v_n \sim \mathcal{N}(0,\Psi)$ and $\varepsilon[z_nv_n^T]=0$.

\begin{deelvraag}
Show that the covariance matrix of the observed data $x_n$ is equal to $\Lambda\Lambda^T + \Psi$.
\tjboxed{
\begin{align*}
\epsilon[x] &= \epsilon[\Lambda z + v] = \Lambda \epsilon[z] + \epsilon[v] = 0\\
\var[x] &= \epsilon[(x-\epsilon[x])(x-\epsilon[x])^T] =  \epsilon[(\Lambda z +v)(\Lambda z +v)^T] \\
    &= \Lambda \epsilon[zz^T] \Lambda^T + \epsilon[vv^T] = \Lambda \Lambda^T + \Psi
\end{align*}
}%end tjboxed    
\end{deelvraag}

\begin{deelvraag} Why is this model not interesting for unconstrained $\Psi$? How does probabilistic PCA handle this problem?
\tjboxed{
(a) Because setting $\Lambda=0$ would result in a `regular' gaussian model with covariance matrix $\Psi$; i.o.w. it's no more interesting than any other gaussian model.\\
(b) If $\Psi$ is diagonal, then all correlations between the components in $x$ \emph{must} be absorbed ('explained') by the rank-$K$ matrix $\Lambda \Lambda^T$. In pPCA, this is achieved by the constraint $\Psi=\sigma^2 I$.
}%end tjboxed    
\end{deelvraag}

\begin{deelvraag}
Which of the following statements are justified? You can pick more than one and read the sign `$\sim$' as: `is similar to'. (Just pick the correct statements; no explanation needed).\\
1: discriminative classification $\sim$ density estimation\\
2: generative classification $\sim$ density estimation\\
3: hidden Markov model $\sim$ factor analysis through time\\
4: Kalman filtering $\sim$ unsupervised regression through time\\
5: clustering $\sim$ supervised classification
\tjboxed{
2 and 4 are correct.
}%end tjboxed    
\end{deelvraag}


%\begin{deelvraag}
%What's the significance of the forward ($\alpha-$) recursion in Linear Dynamical Systems?
%\end{deelvraag}
%
%\begin{deelvraag}
%
%In a particular model with hidden variables, the log-likelihood can be worked out to the following expression:
%$$
%\ell(\theta;D) = \sum_n \log \left(\sum_k \N(x_n|\mu_k,\Sigma_k)\pi_k\right)
%$$
%Is there a problem when trying to do maximum likelihood estimation? Explain shortly.
%\end{deelvraag}
%
%\begin{deelvraag}
%The maximum likelihood estimate (MLE) of the class-conditional mean in a classification problem can be expressed as
%$$
%\hat\mu_k = \frac{\sum_n y_n^k x_n}{\sum_n y_n^k}
%$$
%and the M-step update for the cluster mean in a clustering problem is given by
%$$
%\hat\mu_k = \frac{\sum_n r_n^k x_n}{\sum_n r_n^k}
%$$
%Explain the relation between $y_n^k$ and $r_n^k$.
%\end{deelvraag}

\end{vraag}

% Hier is vraag 2
\begin{vraag}{a) 2 points; b) 2 points; c) 2 points; d) 1 point; e) 1 point; f) 2 points. Total 10 points}


(EM for 2-component Gaussian mixture). Consider an observed IID data set $D=\{x_1,\ldots,x_N\}$ and a proposed model,
\begin{align*}
p(x_n) &= \sum_{k=0}^1 p(x_n,z_n=k|\pi)\\
    &=  p(z_n=1|\pi)p(x_n|z_n=1) + p(z_n=0|\pi)p(x_n|z_n=0)\\
    &= \pi\N_1(x_n) +(1-\pi)\N_0(x_n)
\end{align*}


where we used shorthand notation $\N_k(x_n)\equiv (2\pi\sigma_k^2)^{-1/2}\exp\left(-(x_n-\mu_k)^2/(2\sigma_k^2)\right)$ for the Gaussian distribution. We assume that the parameters $\theta=(\mu_0,\sigma_0^2,\mu_1,\sigma_1^2)$ are known, but the \emph{mixing proportion} parameter $\pi$ is unknown.  The random variable $z_n \in \{0,1\}$ is an \emph{unobserved} `cluster label'.  In this question we will derive an EM-algorithm for maximum likelihood estimation of $\pi$. Let's assume that a estimate $\hat \pi = \pi^{(j)}$ is available from the previous iteration. We will now focus on the $(j+1)$-th iteration in the EM algorithm.

\begin{deelvraag}
Describe shortly the E- and M-steps in the $(j+1)$-th iteration of the EM-algorithm. In particular, complete the following equation set (fill in the stars) for the $(j+1)$-th iteration and shortly describe the meaning of the equations: (Note: the expression $\langle f(x)\rangle_{p(x)}$ stands for the expectation of $f(x)$ w.r.t. probability distribution $p(x)$.)
\begin{align*}
q_n^{(j+1)} &= p(\star|\star) \quad\text{(E-step)}\\
\pi^{(j+1)} &= \arg\max_\pi \langle\star\rangle_{\star} \quad \text{(M-step)}
\end{align*}
\tjboxed{
\begin{align*}
q_n^{(j+1)} &= p(z_n|x_n,\pi^{(j)}) \quad\text{(E-step)}\\
\pi^{(j+1)} &= \arg\max_\pi \langle \sum_n p(x_n,z_n|\pi)\rangle_{q_n^{(j+1)}} \quad \text{(M-step)}
\end{align*}
\textbf{E-step}: $q_n^{(t+1)}$ is the posterior probability of $z_n$, given observation $x_n$ and an estimate $\pi^{(j)}$ from the previous iteration. $q_n$ represents our knowledge about $z_n$. \\
\textbf{M-step}: Maximizes the expected complete-data log-likelihood. Through Jensen's inequality it can be proved that this procedure increases the (observed data) log-likelihood $p(D|\pi)$.
}%end tjboxed    
\end{deelvraag}

%
%\begin{deelvraag}
%Compute the log-likelihood $\ell(\pi)=\log p(D|\pi)$. Why
%\tjboxed{
%\begin{align*}
%\ell(\pi)&=\log p(D|\pi)=\sum_n \log [\pi\N_1(x_n) +(1-\pi)\N_0(x_n)]
%\end{align*}
%}%end tjboxed    
%\end{deelvraag}

\begin{deelvraag}
Work out $p(x_n,z_n=1|\pi)$ (hint: use product rule).  Work out $p(x_n,z_n=0|\pi)$. And now work out the joint distribution $p(x_n,z_n|\pi)$ to a Bernoulli distribution (as in eq.A1, see formula cheat sheet). In this question, you need to work out the probabilities in terms of $z_n$, $\N_0(x_n)$, $\N_1(x_n)$ and $\pi$.
\tjboxed{
$p(x_n,z_n=1) = p(x_n|z_n=1)p(z_n=1)=\pi\N_1(x_n)$\\
$p(x_n,z_n=0) = p(x_n|z_n=0)p(z_n=0)=(1-\pi)\N_0(x_n)$\\
$p(x_n,z_n|\pi) = [\pi\N_1(x_n)]^{z_n}[(1-\pi)\N_0(x_n)]^{1-z_n}$
}%end tjboxed    
\end{deelvraag}

\begin{deelvraag}
Show that the complete-data log-likelihood $\ell_c(\pi) = \sum_n \log p(x_n,z_n|\pi)$ can be worked out to
\begin{equation}
\ell_c(\pi) = \sum_n z_n \log \frac{\pi\N_1(x_n)}{(1-\pi)\N_0(x_n)} + \sum_n \log(1-\pi)\N_0(x_n)
\label{eq:complete-data-log-likelihood}
\end{equation}
\tjboxed{
\begin{align*}
\ell_c(\pi) &= \sum_n \log p(x_n,z_n|\pi)\\
    &=\sum_n \log  \left([\pi\N_1(x_n)]^{z_n}[(1-\pi)\N_0(x_n)]^{1-z_n}\right) \notag\\
    &=\sum_n z_n \log\pi\N_1(x_n) + \sum_n(1-z_n) \log(1-\pi)\N_0(x_n) \notag\\
    &=\sum_n z_n \log \frac{\pi\N_1(x_n)}{(1-\pi)\N_0(x_n)} + \sum_n \log(1-\pi)\N_0(x_n)
\end{align*}
}%end tjboxed    
\end{deelvraag}

To finalize the E-step, we now take the expectation of the complete-data log-likelihood with respect to the posterior distribution $p(z_n|x_n,\pi^{(j)})$. It follows from Eq.1 that we need to compute the expected value of $z_n$.  We'll compute the expected value of $z_n$ in two stages:

%\begin{deelvraag}
%Why do we maximize in the M-step the \emph{expected} complete-data log-likelihood rather than the `regular' complete-data log-likelihood?
%\tjboxed{
%The `regular' complete-data log-likelihood is a function of the unobserved RV $z_n$ and hence cannot be evaluated. The \emph{expected} complete-data log-likelihood can be evaluated (and hence) its maximum can be searched.\\
%Alternative answer: It can be proven through Jensen's inequality that maximizing the \emph{expected} complete-data log-likelihood also maximizes the observed-data log-likelihood.
%}%end tjboxed    
%\end{deelvraag}


\begin{deelvraag}
First show that the expectation $\sum_{\{z_n\}} z_n \cdot p(z_n|x_n,\pi^{(j)})$ can be worked out as follows:
$$
\sum_{\{z_n\}} z_n p(z_n|x_n,\pi^{(j)}) = p(z_n=1|x_n,\pi^{(j)})
$$
\tjboxed{
\begin{align*} \sum_{\{z_n\}}z_np(z_n|x_n,\pi) &= 0\cdot p(z_n=0|x_n,\pi)+1\cdot p(z_n=1|x_n,\pi)\\
    &=p(z_n=1|x_n,\pi)
\end{align*}
}%end tjboxed    
\end{deelvraag}

\begin{deelvraag}
And now use Bayes rule to work out an expression for $p(z_n=1|x_n,\pi^{(j)})$ in terms of $\pi^{(j)}$, $\N_0(x_n)$ and $\N_1(x_n)$.
\tjboxed{
\begin{align*}
p(z_n=1|x_n,\pi^{(j)}) &= \frac{p(x_n|z_n=1)p(z_n=1|\pi^{(j)})}{\sum_k p(x_n|z_n=k)p(z_n=k|\pi^{(j)})}\\
    &=\frac{\pi^{(j)} \N_1(x_n)}{\pi^{(j)} \N_1(x_n) + (1-\pi^{(j)})\N_0(x_n)}
\end{align*}
}%end tjboxed    
\end{deelvraag}

If we use shorthand notation $\zeta_n = p(z_n=1|x_n,\pi^{(j)})$, then the expected complete-data log-likelihood can be written as
\begin{align*}
\langle\ell_c(\pi)\rangle &=\sum_n \zeta_n\log \frac{\pi\N_1(x_n)}{(1-\pi)\N_0(x_n)} + \sum_n \log(1-\pi)\N_0(x_n)
\end{align*}

\begin{deelvraag}
Set $\partial \langle\ell_c(\pi)\rangle / \partial \pi=0$ and obtain an expression for $\pi^{(j+1)}$ in terms of $\pi^{(j)}$, $\N_0(x_n)$ and $\N_1(x_n)$ (i.e. write down the $(j+1)$-th iteration of the M-step).
\tjboxed{
\begin{align*}
\frac{\partial \langle\ell_c(\pi)\rangle}{  \partial \pi} &= \sum_n \frac{\zeta_n}{\pi} + \sum_n \frac{\zeta_n}{1-\pi} - \sum_n \frac{1}{1-\pi}\\
&= \frac{1}{\pi(1-\pi)}\sum_n\left( \zeta_n - n\pi\right)
\end{align*}
Set derivative to zero and it follows that
\begin{align*}
\pi^{(j+1)} &= \frac{1}{N}\sum_n \zeta_n\\
    &= \frac{\pi^{(j)} \N_1(x_n)}{\pi^{(j)} \N_1(x_n) + (1-\pi^{(j)})\N_0(x_n)}
\end{align*}
}%end tjboxed    
\end{deelvraag}
\end{vraag}


%%%%%%%%%%%%%%%%
%%%
%%% MDL vragen
%%%

%%%
%%% Vraag Kolmogorov
%%%
\begin{vraag}{a) 3 points; b) 3 points. Total 6 points}
You observe some data $x^n$. You ask two experts to explain the data.

Expert $A$ uses a data compression system that needs 1537 bits to describe the parameters
of the model and 438 bits to describe the data given the model.

Expert $B$ gives you a system that needs 1325 bits for the parameters and 650 bits for
the data, given the model.
\begin{deelvraag}
  Which expert's result do you prefer?\\
  Explain (briefly) why you select that experts result.
  \tjboxed{%
    The total description length of $A$'s result is $1537+438=1975$ bits.
    For expert $B$ the total description length is $1325+650=1975$ bits. So both experts
    achieve the same complexity. In accordance with Occam's razor I prefer expert
    $B$'s explanation because his/her model is less complex.
  }
\end{deelvraag}
\begin{deelvraag}
  You ask two additional experts.

  Expert $C$ gives you a model with a parameter description length of 1471 bits
  and a data description that needs 450 bits.

  Expert $D$ gives you a model with a parameter description length of 1464 bits
  and a data description that needs 543 bits.

  Of the four experts $A$ to $D$, which result do you prefer, and why?
  \tjboxed{%
    The total complexity for expert $C$ is $1471+450=1921$ bits and for
    expert $D$ it is $1464+543=2007$ bits. Expert $D$ explanation is more
    complex than any of the three others
    so I reject it in accordance with the MDL principle. For the same
    reason I prefer expert $C$'s explanation, because it has the smallest overall
    complexity although the model complexity is larger than for expert $B$.
  }
\end{deelvraag}
\end{vraag}



%%%
%%% Vraag 4 Laplace
%%%
\begin{vraag}{a) 3 points; b) 3 points. Total 6 points}
    Let $X$ be a real valued random variable with probability
    density
    \[ p_X(x) = \frac{e^{-x^2/2}}{\sqrt{2\pi}},\quad\text{for all $x$}. \]
    Also $Y$ is a real valued random variable with conditional
    density
    \[ p_{Y|X}(y|x) = \frac{e^{-(y-x)^2/2}}{\sqrt{2\pi}},\quad\text{for
    all $x$ and $y$}. \]
    \begin{deelvraag}
        Give an (integral) expression for $p_Y(y)$.\\
        Do not try to evaluate the integral.
        \tjboxed{%
            \[ p_Y(y) = \int_{-\infty}^{\infty} p_X(x)p_{Y|X}(y|x)\,dx =
                \int_{-\infty}^{\infty}
                \frac{e^{-\frac12(x^2+(y-x)^2)}}{2\pi}\,dx \]
        }
    \end{deelvraag}
    \begin{deelvraag}
        Approximate $p_Y(y)$ using the Laplace approximation.\\
        Give the detailed derivation, not just the answer.\\
        Hint: You may use the following results.
        \begin{align*}
          \intertext{Let}
          g(x) &= \frac{e^{-x^2/2}}{\sqrt{2\pi}},\quad\text{and} \\
          h(x) &= \frac{e^{-(y-x)^2/2}}{\sqrt{2\pi}},\quad\text{for some real value $y$.} \\
          \intertext{Then}
          \frac{\partial}{\partial x}g(x) &= -xg(x) \\
          \frac{\partial^2}{\partial x^2}g(x) &= (x^2-1)g(x) \\
          \frac{\partial}{\partial x}h(x) &= (y-x)h(x) \\
          \frac{\partial^2}{\partial x^2}h(x) &= ((y-x)^2-1)h(x) \\
        \end{align*}
    \end{deelvraag}
\end{vraag}
%%%
%%% Einde vraag 4: Laplace
%%%

%%% Vraag 5: MDL
%%%

\begin{vraag}{a) 1 point; b) 1 point; c) 2 points; d) 2 points; e) 2 points. Total 8 points}
We implement an e-mail spam filter using two features that we can extract from an e-mail.
A feature can be the occurrence of a particular word or phrase in the e-mail.

Given an e-mail $E$ we denote the extracted features by $F$ and $G$.\\
$F=1$ means that feature $F$ is present in the e-mail $E$.\\
$F=0$ means that feature $F$ is absent. And likewise for feature $G$.\\
The variable $C$ indicates whether $E$ is spam ($C=1$) or not ($C=0$).

We are given 247 e-mails that are already classified. The following table shows how many e-mails contained certain features and the classification.
\begin{center}
  \begin{tabular}{rrr|r}
  $F$ & $G$ & $C$ & nr of e-mails \\
  \hline
  0 & 0 & 0 & 15 \\
  0 & 0 & 1 & 28 \\
  0 & 1 & 0 & 18 \\
  0 & 1 & 1 & 25 \\
  1 & 0 & 0 &  8 \\
  1 & 0 & 1 & 75 \\
  1 & 1 & 0 & 10 \\
  1 & 1 & 1 & 68
  \end{tabular}
\end{center}
\begin{deelvraag}
From the table given above you can determine probability estimates using the maximum likelihood estimates. e.g.\ the probability $P(C=1)$, i.e.\ the probability that an email will be spam, is approximated by:
\[ P(C=1) = \frac{\text{\# of e-mails with }C=1}{\text{total \# of e-mails}} =
   \frac{196}{247}=0.7935.
\]
Note that the method using a beta prior would be better suited but we'll use the maximum likelihood
because it is simpler.

Determine the following estimates.
\begin{gather*}
  P(F=1|C=0), P(F=1|C=1), \\ P(G=1|C=0), P(G=1|C=1), \\
  P(F=0,G=0|C=0), P(F=0,G=1|C=0), \\ P(F=1,G=0|C=0), P(F=1,G=1|C=0), \\
  P(F=0,G=0|C=1), P(F=0,G=1|C=1), \\ P(F=1,G=0|C=1), P(F=1,G=1|C=1).
\end{gather*}
\end{deelvraag}
%
Model $M_1$ for e-mail does not consider any feature. So $P(C)$ can be used to estimate the probability that the next e-mail will be spam or not. We will write that as $P(C|M_1)$.\\
\begin{deelvraag}
  Model $M_2$ considers only feature $F$ to predict whether the next e-mail will be spam or not.
  
  Use Bayes rule and the probability estimates determined in the previous question
  to determine an estimate for $P(C|M_2)=P(C|F)$.
\end{deelvraag}
Model $M_3$ considers feature $G$ only and model $M_4$ considers both $F$ and $G$. Model $M_5$ also considers both $F$ and $G$ but assumes that $F$ and $G$ are independent given the classification $C$.
\begin{deelvraag}
  Use Bayes rule again to show how you would calculate $P(C|M_5)$.
\end{deelvraag}
\begin{deelvraag}
  The models $M_1, M_2,\ldots,M_5$ all have a certain number of free parameters.
  Determine the number of free parameters for each of the five models.
\end{deelvraag}
\begin{deelvraag}
  Given the training set of the 100 e-mail as shown in the table above,
  which of the five models would you prefer? Use an MDL argument in your answer.

  HINT: You will need to calculate an estimate for the email entropy for each model.
  For model $M_1$ you make an estimate of $H(C)$ using the maximum likelihood estimate $P(C=1)=0.7935$.
  Likewise you calculate for $M_2$ the entropy $H(C,F)$ and thus you'll need to compute $P(C,F)$.
  For $M_3$ you must compute the entropy $H(C,G)$; for $M_4$
  you calculate $H(C,F,G)$ and for $M_5$ also $H(C,F,G)$ although this will be a different calculation
  than for $M_4$.
\end{deelvraag}
\end{vraag}

\newpage
\section*{Appendix: formula sheet}

%Consider random variables (RV) $x\in\mathcal{X}$ and $y\in \mathcal{Y}$. The basic axioms of probability theory are the sum and product rules,
%$$
%p(x) + p(\overline{x}) = 1 \quad \text{(sum rule)}
%$$
%$$
%p(x,y) = p(x|y)p(y) \quad \text{(product rule)}
%$$
%
%The following formulas can be derived from the sum and product rules,
%$$
%p(x|y) = \frac{p(x,y)}{p(y)} \quad \text{(conditional probability of $x$, given $y$)}
%$$
%
%$$
%p(x|y) = \frac{p(y|x)p(x)}{p(y)} = \frac{p(y|x)p(x)}{\sum_{x\in\mathcal{X}}p(y|x)p(x)} \quad \text{(Bayes rule)}
%$$
%
%$$
%p(x) = \sum_{y\in\mathcal{Y}} p(x,y) \quad \text{(marginal probability)}
%$$
%
%\medskip
%We use for mean and variance the following notation,
%
%$$
%\varepsilon[x] = \langle x \rangle = \int_x x p(x) \d{x} \quad \text{(expectation)}
%$$
%$$
%\var[x] = \varepsilon[(x-\varepsilon[x])(x-\varepsilon[x])^T] \quad \text{(variance)}
%$$
%\medskip
%\textbf{Some probability distributions}\\
The \emph{Bernoulli distribution} is a discrete distribution having two possible outcomes labeled by $x = 0$ and $x = 1$ in which $x = 1$ ("success") occurs with probability $\theta$ and $x = 0$ ("failure") occurs with probability $1-\theta$. It therefore has probability function
\begin{equation}
p(x|\theta) =\theta^x(1-\theta)^{1-x}
\tag{A.1}
\end{equation}

The \emph{Gaussian distribution} with mean $\mu$ and variance $\sigma^2$ is defined as
$$
\N(x|\mu,\sigma^2) = \frac{1}{\sqrt{2\pi}\sigma} \exp\left\{-\frac{1}{2\sigma^2}(x-\mu)^2\right\}
$$
\end{exam}
\end{document}
