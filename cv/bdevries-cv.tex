% Copyright 2013 Christophe-Marie Duquesne <chmd@chmd.fr>
% Copyright 2014 Mark Szepieniec <http://github.com/mszep>
% 
% ConText style for making a resume with pandoc. Inspired by moderncv.
% 
% This CSS document is delivered to you under the CC BY-SA 3.0 License.
% https://creativecommons.org/licenses/by-sa/3.0/deed.en_US

\startmode[*mkii]
  \enableregime[utf-8]  
  \setupcolors[state=start]
\stopmode

\setupcolor[hex]
\definecolor[titlegrey][h=757575]
\definecolor[sectioncolor][h=397249]
\definecolor[rulecolor][h=9cb770]

% Enable hyperlinks
\setupinteraction[state=start, color=sectioncolor]

\setuppapersize [A4][A4]
\setuplayout    [width=middle, height=middle,
                 backspace=20mm, cutspace=0mm,
                 topspace=10mm, bottomspace=20mm,
                 header=0mm, footer=0mm]

%\setuppagenumbering[location={footer,center}]

\setupbodyfont[11pt, helvetica]

\setupwhitespace[medium]

\setupblackrules[width=31mm, color=rulecolor]

\setuphead[chapter]      [style=\tfd]
\setuphead[section]      [style=\tfd\bf, color=titlegrey, align=middle]
\setuphead[subsection]   [style=\tfb\bf, color=sectioncolor, align=right,
                          before={\leavevmode\blackrule\hspace}]
\setuphead[subsubsection][style=\bf]

\setuphead[chapter, section, subsection, subsubsection][number=no]

%\setupdescriptions[width=10mm]

\definedescription
  [description]
  [headstyle=bold, style=normal,
   location=hanging, width=18mm, distance=14mm, margin=0cm]

\setupitemize[autointro, packed]    % prevent orphan list intro
\setupitemize[indentnext=no]

\setupfloat[figure][default={here,nonumber}]
\setupfloat[table][default={here,nonumber}]

\setuptables[textwidth=max, HL=none]

\setupthinrules[width=15em] % width of horizontal rules

\setupdelimitedtext
  [blockquote]
  [before={\setupalign[middle]},
   indentnext=no,
  ]


\starttext
\startalignment[center]
  \blank[2*big]
  {\tfd Bert de Vries}
  \blank[3*medium]
\stopalignment

\thinrule

\startblockquote
GN ReSound & Eindhoven University of Technology\crlf
Het Eeuwsel 6\crlf
5612 AS Eindhoven, the Netherlands\crlf
tel. +31-6-1922-2046\crlf
email bdevries {\em at} ieee {\em dot} org\crlf
web \useURL[url1][http://bertdv.nl]\from[url1]

Version: Dec. 2016
\stopblockquote

\thinrule

\subsection[principal-interests]{Principal Interests}

Signal processing, machine learning, computational neuroscience, data
science, biomedical engineering, research management, technical writing;
applications to multimedia processing, medical devices, hearing
rehabilitation and clinical trial design/analysis.

\subsection[academic-background]{Academic Background}

\startdescription{1999}
  {\bf Ph.D.~Electrical Engineering},
  \useURL[url2][http://www.ece.ufl.edu/][][University of
  Florida]\from[url2], Gainesville, FL

  Ph.D.~research in signal processing under direction of Professor Jose
  C. Principe. Dissertation title: {\em Temporal processing with neural
  networks--the development of the gamma model}.
\stopdescription

\startdescription{1986}
  {\bf M.Sc. Electrical Engineering},
  \useURL[url3][http://tue.nl][][Eindhoven University of
  Technology]\from[url3], Eindhoven, the Netherlands

  Focus areas: medical engineering (thesis: intelligent alarms during
  anaesthesia) and digital communications.
\stopdescription

\subsection[employment-history]{Employment History}

\startdescription{2012-pres.}
  {\bf Professor}, \useURL[url4][http://tue.nl][][Eindhoven University
  of Technology]\from[url4],
  \useURL[url5][http://www.sps.ele.tue.nl/][][Signal Processing Systems
  Group]\from[url5] (EE dept.), Eindhoven, the Netherlands

  \startitemize[packed]
  \item
    1 day/week appointment; previous engagement: Research Fellow
    ('04-'11)
  \item
    Research on {\em Personalization of Medical Signal Processing
    Systems}
  \item
    Teach graduate class on
    \useURL[url6][http://www.sps.ele.tue.nl/members/b.vries/teaching/5mb20/index.html][][Adaptive
    Information Processing]\from[url6]
  \item
    Inaugural lecture:
    \useURL[url7][./files/Bert\%20de\%20Vries\%20-\%2013sep2013\%20-\%20In\%20situ\%20personalization\%20of\%20signal\%20processing\%20systems\%20-\%20\%20inaugural\%20lecture\%20booklet\%20\%28final\%20version\%29.pdf][][{\em In
    Situ Personalization of Signal Processing Systems}]\from[url7] (at
    \useURL[url8][http://goo.gl/EoU0SE][][youtube]\from[url8]), Sep.
    2013
  \stopitemize
\stopdescription

\startdescription{1999-pres.}
  {\bf Principal Scientist},
  \useURL[url9][http://www.gnresound.dk/][][GN ReSound]\from[url9]
  (Philips Hearing Technologies until 2001), Eindhoven, the Netherlands

  \startitemize[packed]
  \item
    Other engagements include: DSP Functional Leader ('11-pres.), Head
    DSP Research ('08-'11), Manager External Research ('01-'08),
    Technology Leader ('99-'01, Philips)
  \item
    Research PI on low-power signal processing technology for the next
    generation of digital hearing aids
  \item
    Leadership/management tasks include(d) all aspects of team and
    project management (teams of about 10 engineers); (responsible for)
    the corporate DSP research track, including the roadmap, budget and
    management; initiating and managing key studies at academic
    institutions and contract research organizations
  \stopitemize
\stopdescription

\startdescription{1992-'99}
  {\bf Member Technical Staff},
  \useURL[url10][http://www.sarnoff.com/][][Sarnoff
  Corporation]\from[url10] (today
  \useURL[url11][http://www.sri.com][][SRI Int'l]\from[url11]),
  Princeton, NJ

  \startitemize[packed]
  \item
    Previous engagement: Postdoctoral fellow ('92-'93)
  \item
    Research in advanced signal processing algorithms, initiating new
    technical and commercial thrusts, technical proposal writing and
    project management
  \item
    Principal investigator of funded projects on keyword spotting,
    digital hearing aids signal processing, speech enhancement and
    noise-robust speech recognition (co-PI)
  \item
    Co-initiated and developed signal processing in financial markets
    program at Sarnoff
  \item
    Member medical image processing research team. Funded projects
    include blind signal processing for breast mammography and
    perceptually optimized image coding
  \stopitemize
\stopdescription

\startdescription{1987-'91}
  {\bf Research/Teaching Assistant},
  \useURL[url12][http://www.ufl.edu/][][University of
  Florida]\from[url12], Gainesville, FL

  \startitemize[packed]
  \item
    Taught and assisted in graduate classes in digital signal
    processing, control theory and computer architecture.
  \stopitemize
\stopdescription

\subsection[special-achievements]{Special Achievements}

\subsubsection[awards]{Awards}

\startitemize
\item
  {\em Return-on-Performance Award}, for \quotation{technical work on
  Speech Enhancement technology}, Sarnoff Corporation, 1998
\item
  {\em David Sarnoff Achievement Award}, for \quotation{leadership and
  technical contributions in the area of adaptive speech enhancement},
  Sarnoff Corporation, 1997
\item
  {\em David Sarnoff Event Focus Award} for \quotation{Winning Sarnoff's
  First Commercial Contract for Speech Processing}, David Sarnoff
  Research Center, 1996
\item
  {\em Presidential Recognition Award}, University of Florida, 1988
\item
  {\em δ-Butterweck Award} (awards top GPA), Technical University
  Eindhoven, 1984
\stopitemize

\subsubsection[invited-lectures-selection]{Invited Lectures (selection)}

\startitemize
\item
  University College London (UCL), \quotation{A Factor Graph Approach to
  Active Inference}, Nov. 2016
\item
  Keynote lecture on \quotation{The Future of Hearing Aid
  personalization}, Cochlear/ReSound Event, Sep.2016
\item
  WIC Mid-winter meeting on \quote{Big Data and Data Analytics},
  \quotation{Design of Signal Processing Algorithms through
  Probabilistic Inference}, Eindhoven, February 2016
\item
  \useURL[url13][http://www.cqm.nl/][][CQM]\from[url13], \quotation{In
  Situ Machine Learning for Signal Processing Systems}, Eindhoven,
  August 2015
\item
  Radboud University Nijmegen, \quotation{Probabilistic Hearing Loss
  Compensation}, Nijmegen, March 2015
\item
  INCAS3 Institute, \quotation{In Situ Personalization of Signal
  Processing Systems}, Assen, October 2014
\item
  Leiden University Medical Center, New Year's keynote lecture on
  \quotation{Personalization of Medical Signal Processing Systems},
  Leiden, January 2014
\item
  Int'l Symposium on Auditory and Audiological Research (ISAAR),
  \quotation{Is Hearing Aid Signal Processing ready for Machine
  Learning?}, Nyborg (DK), August 2013
\item
  Clinical Physicist Post-graduate school,\quote{'The Future of Hearing
  Aids}', Amersfoort January 2013
\item
  Delft Univ. of Technology, \quote{'Machine Learning for Hearing Aids
  Technology}', Delft March 2012
\item
  International Forum for Hearing Instrument Developers,
  \quote{'Bayesian Machine Learning for Hearing Aid Design, Fitting and
  Personalization}', Oldenburg (Germany), June 2011
\item
  University of Florida, \quote{'Machine Learning Trends in the Hearing
  Aids Industry}', Gainesville, FL, April 2010
\item
  SIKS Research School, \quote{'Gaussian mixture models and the EM
  Algorithm}', Vught, NL, Dec 2008
\item
  GN Nordic Audiology College, \quote{'Learning technology in hearing
  aids}', Oslo, Norway, Sep 29, 2006
\item
  University of Nijmegen, \quote{'Machine learning for hearing aids}',
  Nijmegen, Netherlands, June 2004
\item
  University of Florida, \quote{'DSP for modern industrial hearing
  aids}', Gainesville, FL, January 2004
\item
  International Forum for Hearing Aid Developers,
  \quote{'Warped-frequency filterbanks}', Oldenburg, Germany, July 2003
\item
  Keynote address \quote{'An industrial perspective on intelligent
  hearing aids}' at 2nd McMaster-Gennum Workshop on Intelligent Hearing
  Instruments, Niagara-on-the-Lake, ON, Sep 2001
\item
  NIDCA/NASA/VA Hearing Aids Improvement Conference, May 1997
\item
  Lucent Technologies, Bell Laboratories, November 1996
\item
  AT&T Research, Murray Hill, NJ, July 1996
\item
  NSA (U.S. Government), June 1993
\item
  Neural Network Workshop, Rutgers University, October 1992
\item
  David Sarnoff Research Center, October 1991
\stopitemize

\subsubsection[professional-activities-selection]{Professional
Activities (selection)}

\startitemize
\item
  2012 - pres., {\bf Associate Editor} for
  \useURL[url14][http://tnsre.embs.org/][][IEEE Transactions on Neural
  Systems and Rehabilitation Engineering]\from[url14]
\item
  2012 - 2015, Invited member annual European Mathworks Advisory Board
  meetings
\item
  2010, Invited jury member for Open Technology Program (OTP) research
  proposals to Dutch Technology Foundation (STW)
\item
  11/2005 and 5/2006, Invited DSP expert on IWT (Flemish Institute for
  Science and Technology) panel to evaluate candidate PhD proposals,
  Brussels
\item
  5/2002, Organizer/chair special session `DSP for Intelligent Hearing
  Aids', ICASSP 2002, Orlando, FL
\item
  1997-'98, Publicity chair, Neural Networks for Signal Processing
  Workshop, Amelia island, Florida (1997) and Cambridge, UK (1998)
\item
  1996 and 1998, Session chair Non-linear Systems Identification,
  ICASSP-96, Atlanta, GA (1996) and IEEE NNSP-98 Workshop, Cambridge, UK
  (1998)
\item
  1995-\quote{98, (Elected) member of}\quote{IEEE Technical Committee on
  Neural Networks for Signal Processing Society}'
\item
  1993, Invited researcher in government sponsored \quote{'Robust Speech
  Processing Workshop}'
\item
  1986 - pres., Member of various professional societies (e.g.~IEEE,
  INNS)
\stopitemize

\subsubsection[refereed-publications]{Refereed Publications}

IEEE Transactions on Signal Processing, IEEE Transactions on Neural
Networks, NeuroComputing Journal, Neural Networks Journal, EURASIP
Journal of Applied Signal Processing, Advances in Neural Information
Processing Systems (NIPS) Conferences, ICASSP Conferences and others

\subsection[activities-at-eindhoven-univ.-of-technology-tue]{Activities
at Eindhoven Univ. of Technology (TU/e)}

\subsubsection[research-funding]{Research Funding}

Research at TU/e focusses on applications of Bayesian machine learning
to personalization of hearing aid algorithms.

\startitemize
\item
  2017 - 2021, together with dr.
  \useURL[url15][http://www.es.ele.tue.nl/~sander/][][Sander
  Stuijk]\from[url15] and prof.
  \useURL[url16][http://www.es.ele.tue.nl/~heco/][][Henk
  Corporaal]\from[url16], {\bf \lettertilde{}550K euro} supporting 3 PhD
  students, from \useURL[url17][http://www.stw.nl/en][][STW]\from[url17]
  to pursue research on \useURL[url18][][][Autonomous Acoustic
  Systems]\from[url18] in the context of
  \useURL[url19][http://www.stw.nl/en/node/8025][][energy-autonomous
  systems for IoT]\from[url19].
\item
  2015 - 2019, together with dr.
  \useURL[url20][http://www.sps.ele.tue.nl/members/T.J.Tjalkens/][][Tjalling
  Tjalkens]\from[url20], {\bf \lettertilde{}500K euro} supporting 2 PhD
  students, from Dutch Technology Foundation
  \useURL[url21][http://www.stw.nl/en/][][STW]\from[url21] to pursue
  research on
  \useURL[url22][http://stw.nl/nl/content/hearscan-towards-data-driven-hearing-aids][][Data-driven
  Hearing Aids]\from[url22].
\item
  2014 - 2018, {\bf \lettertilde{}500K euro} supporting 2 PhD students
  at TU/e, from GN ReSound to support research on hearing aids
  personalization.
\item
  2006 - 2008, {\bf \lettertilde{}130K euro} from GN ReSound to support
  2 PDEng students at TU/e.
\item
  2006 - 2010, together with
  \useURL[url23][http://www.cs.ru.nl/staff/Tom.Heskes][][Tom
  Heskes]\from[url23] and
  \useURL[url24][http://www.ac-amc.nl/medewerkers/dreschler.html][][Wouter
  Dreschler]\from[url24], {\bf \lettertilde{}650K euro} grant from
  \useURL[url25][http://www.stw.nl][][STW]\from[url25] to pursue further
  research on
  \useURL[url26][http://www.nwo.nl/en/research-and-results/research-projects/35/2300148635.html][][Personalization
  of Hearing Aids through Bayesian Preference Elicitation]\from[url26].
\stopitemize

\subsubsection[teaching]{Teaching}

\startitemize
\item
  {\bf 2005 - pres.},
  \useURL[url27][http://bertdv.github.io/teaching/AIP-5SSB0/][][{\bf Adaptive
  Information Processing}]\from[url27]. Together with
  \useURL[url28][http://www.sps.ele.tue.nl/members/T.J.Tjalkens][][dr.ir.
  Tjalling Tjalkens]\from[url28], since spring 2005 I teach a core
  graduate class on the fundamentals of machine learning.
\item
  {\bf 2011 - pres.}, guest lecturer for EE course {\bf Development of
  (Electro)-technology} at TU/e
\item
  2004, Machine Learning. I organized a machine learning class (at GN
  ReSound premises) for TU/e graduate students and GN ReSound staff.
\stopitemize

\subsubsection[student-supervision]{Student Supervision}

\startitemize[n][stopper=.,width=2.0em]
\item
  6/2016, Wouter van Roosmalen, M.Sc. thesis project, {\em In-situ
  Design of Noise Reduction Algorithms}
\item
  6/2016, Anouk van Diepen, M.Sc. internship, {\em Derivation and
  Implementation of Gausssian Mixture Model in a Forney-style Factor
  Graph}
\item
  10/2015, Pradeep Kumar, M.Sc. practical training project, {\em On
  Discrete-Valued Message Passing in Factor Graphs}
\item
  10/2014, Rene Duijkers, M.Sc. thesis project, {\em A Factor Graph
  Approach to Hearing Loss Compensation}
\item
  10/2014, Max Schoonderbeek, M.Sc. thesis project, {\em A Factor Graph
  Approach to Gaussian Process Preference Learning}
\item
  6/2014, Art Senders, M.Sc. practical training project, {\em A Julia
  Toolbox for Forney-style Factor Graphs}
\item
  4/2014, Robert Leenders, B.Sc. final project, {\em Gaussian Process
  based Preference Learning as a Classification Problem}
\item
  1/2014, Rene Duijkers, M.Sc. practical training project,
  \useURL[url29][./files/Duijkers\%20-\%202014\%20-\%20Bayesian\%20Online\%20Spectral\%20Tracking.pdf][][{\em Online
  Bayesian Spectral Tracking}]\from[url29]
\item
  12/2013, Brian Hutama Susilo, M.Sc. practical training project,
  {\em Automated Tuning Algorithm for Low-latency PC-based Audio
  Processing}
\item
  8/2013, Zijian Xu, M.Sc. thesis project,
  \useURL[url30][./files/ZijianXu\%20-\%202013\%20-\%20Fast\%20Design\%20of\%20Audio\%20Processing\%20Algorithms\%20by\%20Interactive\%20Parameter\%20Exploration.pdf][][{\em Fast
  Design of Audio Processing Algorithms by Interactive Parameter
  Exploration}]\from[url30]
\item
  8/2013, Timur Bagautdinov, M.Sc. thesis project,
  \useURL[url31][./files/Bagautdinov\%20-\%202013\%20-\%20A\%20Machine\%20Learning\%20Framework\%20for\%20Bayesian\%20Signal\%20Processing.pdf][][{\em A
  Machine Learning Framework for Bayesian Signal
  Processing}]\from[url31]
\item
  6/2013 Marno van der Maas, B.Sc. research project, {\em Browser-based
  Remote Control of Hearing Aids}
\item
  12/2012 Timur Bagautdinov, traineeship project,
  \useURL[url32][./files/Bagautdinov\%20-\%202012\%20-\%20A\%20MATLAB-C\%2B\%2B\%20toolbox\%20for\%20Factor\%20Graph\%20Modeling.pdf][][{\em A
  MATLAB/C++ toolbox for Factor Graph Modeling}]\from[url32]
\item
  6/2012, Maarten Thomassen, M.Sc. practical training project,
  {\em Spectral Audio Monitoring}
\item
  4/2012, Joris Kraak, M.Sc.-thesis,
  \useURL[url33][./files/Kraak\%20-\%202012\%20-\%20Computer-Aided\%20Algorithm\%20Design\%20for\%20Audio\%20Processing.pdf][][{\em Computer-Aided
  Algorithm Design for Audio Processing}]\from[url33]
\item
  10/2010, Joris Kraak, M.Sc. practical training project,
  {\em Optimization of a Spectral Noise Tracking Algorithm}
\item
  8/2010, Jianfeng Li, M.Sc.-thesis,
  \useURL[url34][./files/Li\%20-\%202010\%20-\%20acoustic-scene-adaptive-speech-enhancement.pdf][][{\em Acoustic
  scene-adaptive speech enhancement}]\from[url34]
\item
  8/2009, Jianfeng Li, M.Sc.-project, {\em Spatial defect clustering on
  semiconductor wafers using image processing techniques}
\item
  9/2008, Xueru Zhang, P.D.Eng.-thesis, {\em Bayesian periodogram
  smoothing for speech enhancement}
\item
  6/2008, Rene Besseling, M.Sc.-project, {\em Gaussian processes in
  Bekesy audiometry}
\item
  8/2007, Serkan Ozer, M.Sc.-thesis, {\em Bayesian linear regression for
  user-adaptive hearing aids}
\item
  6/2007, Ronnie van Loon, M.Sc.-thesis, {\em a Probabilistic Approach
  to Sound Classification}
\item
  9/2006, Anton Vakrushev, P.D.Eng.-thesis, {\em Interactive machine
  learning for Personalization of hearing aid algorithms}
\item
  10/2005, Jorik Caljouw, M.Sc. practical training on {\em PDA-based
  Interfacing to a real-time audio platform}
\item
  10/2005, Paul Aelen, M.Sc. project, {\em Determination of the
  Intra-Uterine Pressure with electrodes on the abdomen}
\item
  6/2005, Job Geurts, M.Sc. practical training on {\em A PC-based
  real-time simulation platform for evaluating hearing aid algorithms}
\stopitemize

\subsubsection[member-of-ph.d.committee]{Member of Ph.D.~Committee}

\startitemize[n][stopper=.]
\item
  01/2017, Math Verstraelen, Ph.D., The WaveCore - A Scalable
  Architecture for Real-time Audio Procesing, University of Twente.
\item
  12/2016, Amir Jalalirad, Ph.D., Supervised Learning through
  Feature-based Models, TU Eindhoven
\item
  6/2015, Yuan Zeng, Ph.D., Distributed Speech Enhancement in Wireless
  Acoustic Sensor Networks, TU Delft
\item
  12/2013, Ingeborg Brons, Ph.D., Perceptual evaluation of noise
  reduction in hearing aids, University of Amsterdam
\item
  9/2013, Jelte Vink, Ph.D., Machine Learning for Efficient Object
  Recognition, TU Eindhoven
\item
  10/2011, Adriana Birlutiu, Ph.D., Machine Learning for Pairwise Data,
  University of Nijmegen
\stopitemize

\subsection[publications]{Publications}

\startitemize[packed]
\item
  Publications with more than 50 citations at January 2016 are indicated
  by {\bf {[}\#citations{]}}. See also my
  \useURL[url29][http://scholar.google.nl/citations?user=x3EIIHEAAAAJ][][google
  scholar page]\from[url29].
\stopitemize

\subsubsection[journal-articles-and-book-chapters]{Journal Articles and
Book Chapters}

\startitemize[n][stopper=.,width=2.0em]
\item
  Thijs van de Laar and Bert de Vries,
  \useURL[url30][http://arxiv.org/abs/1602.01345][][A Probabilistic
  Modeling Approach to Hearing Loss Compensation]\from[url30], {\em IEEE
  Tr. on Audio, Speech and Language Processing}, Nov. 2016
\item
  Rik Vullings et al., An Adaptive Kalman Filter for ECG Signal
  Enhancement, {\em IEEE Transactions on Biomedical Engineering},
  vol.58, no.4, April 2011 {\bf {[}58{]}}
\item
  A. Ypma et al.,
  \useURL[url31][http://www.hindawi.com/GetArticle.aspx?doi=10.1155/2008/183456][][On-line
  Personalization of Hearing Instruments]\from[url31], {\em EURASIP
  Journal on Audio, Speech, and Music Processing}, September 2008
\item
  Tjeerd Dijkstra et al.,
  \useURL[url32][http://www.hearingreview.com/issues/articles/2007-10_05.asp][][The
  Learning Hearing Aid: Common-Sense Reasoning in Hearing Aid
  Circuits]\from[url32], The Hearing Review, issue October 2007
\item
  David Zhao et al., On-line Noise Estimation Using Stochastic-Gain HMM
  for Speech Enhancement, {\em IEEE Transactions on Audio, Speech and
  Language Processing}, vol.16, no.4, May 2008
\item
  Jose Principe et al., Locally Recurrent Networks: The Gamma Operator,
  Properties and Extensions, invited book chapter in {\em Neural
  Networks and Pattern Recognition}, Omidvar and Dayhoff (eds.),
  Academic Press, 1997
\item
  Bert de Vries, Short term memory structures for dynamic neural
  networks, book chapter in: {\em Artificial Neural Networks for Speech
  and Vision}, Richard Mammone (ed.), Chapman & Hall Ltd., 1994
\item
  Bert de Vries and Jose Principe, The gamma model--A new neural network
  for temporal processing, {\em Neural Networks} vol.~5(4), pp.~565-576,
  1992 {\bf {[}240{]}}
\item
  Jose Principe and Bert de Vries, The gamma filter--A new class of
  adaptive IIR filters with restricted feedback, _ IEEE transactions on
  signal processing_ vol.~41(2), pp.~649-656, 1992 {\bf {[}142{]}}
\item
  Bert de Vries,
  \useURL[url33][http://ufdc.ufl.edu/UF00082173/00001][][Temporal
  processing with neural networks-the development of the Gamma
  model]\from[url33], {\em Ph.D.~dissertation}, University of Florida,
  1991
\item
  Joachim Gravenstein et al., Sampling intervals for clinical monitoring
  of variables during anesthesia, {\em Journal of clinical monitoring}
  vol 5(1), 1989
\item
  Jan J. van der Aa, Bert de Vries and Joachim Gravenstein, Toward more
  sophisticated monitoring alarms, {\em Journal of clinical monitoring}
  4 (2), 1986
\stopitemize

\subsubsection[patents]{Patents}

\startitemize[n][stopper=.,width=2.0em]
\item
  Almer van den Berg and Bert de Vries, Sound signal modelling based on
  recorded object sound, filed by GN ReSound, EP16206941.3, Dec. 2016
\item
  Bert de Vries and Joris Kraak, Automated Scanning for Hearing Aid
  Parameters, filed by GN ReSound, July 2016
\item
  Fredrik Gran et al., Performance-based In Situ Optimization of Hearing
  Aids, filed by GN ReSound, with the Danish Patent and Trademark
  Office, PA 2015-70379, June 2015
\item
  Bert de Vries and Erik van der Werf, A Multi-band Signal Processor for
  Digital Audio Signals, filed by GN ReSound with European Patent and
  Office, Feb. 2014
\item
  Andrew Dittberner et al., A Location Learning Hearing Aid, filed by GN
  ReSound with European Patent and Office, App. 13197214.3-1901,
  December 2013
\item
  Bert de Vries and Mojtaba Farmani, A Hearing Aid with Probabilistic
  Hearing Loss Compensation, filed by GN ReSound with US Patent and
  Trademark Office, App. number 14077031, Nov. 2013
\item
  Bert de Vries et al., Efficient evaluation of hearing ability,
  submitted by GNR Ref.: P1669 EP, Albihns Ref.: P13304 US / P13303,
  April 2009
\item
  Alexander Ypma et al., Asymmetric synchronization of hearing aid
  algorithms, submitted by GN ReSound, patent no. 09174982.0-2225, filed
  4-Nov-2009
\item
  Alexander Ypma et al., Learning control of hearing aid parameter
  settings, submitted by GN ReSound, filed 16-Mar-2007
\item
  Bert de Vries and Alexander Ypma, Optimization of Hearing Aid
  Parameters, filed by GN ReSound, patent no. WO/2007/042043, 10/13/06
\item
  Bert de Vries, Bastiaan Kleijn, Alexander Ypma and David Zhao, Method
  and Apparatus for Improved Estimation of Non-stationary Noise for
  Speech Enhancement, filed by GN ReSound, patent no. 06119399.1-224,
  08/23/06
\item
  Bert de Vries and Rob de Vries, Fitting methodology and hearing
  prosthesis based on signal-to-noise ratio loss data, USA patent
  registered for GN ReSound, no. 20040047474, 03/11/2004
\item
  L. Parra and B. de Vries, Method and apparatus for adaptive speech
  detection by applying a probabilistic description to the
  classification and tracking of signal components, patent registered
  for Sarnoff Corporation, LG Electronics, Inc., no. 6691087, 10-Feb.
  2004
\item
  Bert de Vries, Noise Spectrum Tracking for Speech Enhancement, patent
  registered for Sarnoff Corporation, no. US6289309, 9/11/2001
  {\bf {[}71{]}}
\item
  J. Lubin et al., Method and apparatus for training a neural network to
  learn and use fidelity metric as a control mechanism, patent
  registered for Sarnoff Corporation, no. US6075884, 6/13/2000
\item
  Bert de Vries, Method and apparatus for filtering signals using a
  gamma delay line based estimation of power spectrum, patent registered
  for Sarnoff Corporation, no. US6073152, 6/6/2000
\item
  M. Brill, J. Lubin, B. de Vries, O. Finard, Method and apparatus for
  assessing the visibility of differences between two image sequences,
  patent registered for Sarnoff Corporation, no. US5974159, 10/26/1999
  {\bf {[}76{]}}
\item
  Bert de Vries, Method and system for training a neural network with
  adaptive weight updating and adaptive pruning in principal components
  space, patent registered for David Sarnoff Research Center, no.
  5,812,992, 9/22/98
\item
  Bert de Vries and Jose Principe, An adaptive filter based on a
  recursive delay line, patent registered for University of Florida, no.
  5,301,135, April 1994
\stopitemize

\subsubsection[professional-interviews]{Professional Interviews}

\startitemize[n,packed][stopper=.]
\item
  \useURL[url30][http://www.audiology-worldnews.com/focus-on/1215-introducing-data-science-hearing-aids-on-the-brink-of-a-paradigm-shift][][Introducing
  Data Science: Hearing Aids on the Brink of a Paradigm
  Shift]\from[url30]. Interview in
  \useURL[url31][http://www.audiology-worldnews.com/][][Audiology Info
  Magazine]\from[url31], Dec 2014
\stopitemize

\subsubsection[conferences-and-workshops]{Conferences and Workshops}

\startitemize[n][stopper=.,width=2.0em]
\item
  Mojtaba Farmani and Bert de Vries, A Probabilistic Approach To Hearing
  Loss Compensation, {\em IEEE Machine Learning for Signal Processing
  workshop} (MLSP), Reims, FR, Sep 2014
\item
  Bert de Vries et al., Efficient Hearing Aid Spectral Signal Processing
  with an Asynchronous Warped Filterbank, {\em Int'l Hearing Aid
  Research Conference} (IHCON), Lake Tahoe, CA, August 2014
\item
  Bert de Vries and Andrew Dittberner, Is Hearing Aid Signal Processing
  Ready for Machine Learning? {\em Int'l Symposium on Auditory and
  Audiological Research}, Nyborg, DK, Aug. 2013
\item
  Ungureanu C. et al., A Bayesian Network for Detection of Seizures,
  {\em 1st Jan Beneken Conference on Modeling and Simulation of Human
  Physiology}, Eindhoven, NL, 2013
\item
  Petkov P. et al., Discrete Choice Models for Non-Intrusive Quality
  Assessment, {\em Interspeech 2011}, Florence, Italy, 2011
\item
  Rob de Vries et al., A software suite for automatic beamforming
  calibration, {\em Int'l Hearing Aid Research Conference}, Lake Tahoe,
  CA, August 2010
\item
  S.I. Mossavat et al., A Bayesian hierarchical mixture of experts
  approach to estimate speech quality, {\em QoMEX 2010}, Trondheim,
  Norway, June 2010
\item
  Jos Leenen and Bert de Vries, Current DSP and Machine Learning Trends
  in the Hearing Aids Industry, {\em IEEE Benelux Signal Processing
  Symposium: Signal Processing for Digital Hearing Aids}, Delft, NL,
  April 2010
\item
  Xueru Zhang et al., Bayesian periodogram smoothing for speech
  enhancement, {\em European Symposium on Artificial Neural Networks
  (ESANN-09)}, Bruges, April 2009
\item
  Adriana Birlutiu et al., Towards hearing aid personalization:
  preference elicitation from audiological data, {\em Scientific
  ICT-Research Event Netherlands (SIREN)}, Amsterdam, Sep. 2008
\item
  Tjeerd Dijkstra et al., HearClip: an Application of Bayesian Machine
  Learning to Personalization of Hearing Aids, Presentation at
  {\em Dutch Society for Audiology Meeting}, Sep. 2008
\item
  Bert de Vries, Fast Model-Based Fitting through Active Data Selection,
  {\em Int'l Hearing Aid Research Conference}, Lake Tahoe, CA, August
  2008
\item
  Rolph Houben et al., Construction of a virtual subject response
  database to reduce subject testing, {\em Int'l Hearing Aid Research
  Conference}, Lake Tahoe, CA, August 2008
\item
  Bert de Vries et al., The Complexity of Hearing Aid Fitting, presented
  at {\em International Symposium on Auditory and Audiological Research
  2007}, Helsingor, Denmark, August 2007
\item
  Jos Leenen et al., Learning Volume Control for Hearing Aids, presented
  at {\em International Symposium on Auditory and Audiological Research
  2007}, Helsingor, Denmark, August 2007
\item
  Alexander Ypma et al., Bayesian Feature Selection for Hearing Aid
  Personalization, {\em MLSP-07}, Thessaloniki, Greece, 2007
\item
  Adriana Birlutiu et al., Personalization of Hearing Aids through
  Bayesian Preference Elicitation, {\em NIPS workshop on User Adaptive
  Systems}, Whistler, BC, Canada, December 2006
\item
  Bert de Vries et al., Bayesian Machine Learning for Personalization of
  Hearing Aid Algorithms, {\em Int'l Hearing Aid Research Conference},
  Lake Tahoe, CA, August 2006
\item
  Alexander Ypma, Bert de Vries and Job Geurts, Robust Volume Control
  Personalization from On-line Preference Feedback, {\em IEEE Int.
  Workshop on Machine Learning for Signal Processing}, Maynooth,
  Ireland, 2006
\item
  Bert de Vries, Tom M. Heskes and Tjeerd M. H. Dijkstra, Bayesian
  Incremental Utility Elicitation with Application to Hearing Aids
  Personalization, {\em Valencia/ISBA 8th World Meeting on Bayesian
  Statistics}, Benidorm, Spain, June 2006
\item
  Tjeerd M. H. Dijkstra et al., A Bayesian decision-theoretic framework
  for psychophysics, {\em Valencia/ISBA 8th World Meeting on Bayesian
  Statistics}, Benidorm, Spain, June 2006
\item
  Alexander Ypma, Bert de Vries and Job Geurts, A learning volume
  control that is robust to user inconsistency, {\em The second annual
  IEEE BENELUX/DSP Valley Signal Processing Symposium}, Antwerp, March
  2006
\item
  Paul Aelen et al., Electrohysterographic Estimation of the
  Intra-Uterine Pressure, {\em The second annual IEEE BENELUX/DSP Valley
  Signal Processing Symposium}, Antwerp, March 2006
\item
  Tom Heskes and Bert de Vries, Incremental Utility Elicitation for
  Adaptive Personalization, {\em The 17th Belgian-Dutch Conference on
  Artificial Intelligence}, Brussels, Belgium, October 2005
\item
  Bert de Vries and Rob de Vries, An Integrated Approach to Hearing Aid
  Algorithm Design, {\em Int'l Hearing Aid Research Conference}, Lake
  Tahoe, CA, August 2004
\item
  Harald Pobloth et al., Speech Coding for Wireless Communication in the
  Hearing Aid Environment, {\em Int'l Hearing Aid Research Conference},
  Lake Tahoe, CA, August 2004
\item
  Bert de Vries and Rob de Vries,An Integrated Approach to Hearing Aid
  Algorithm Design for Enhancement of Audibilit y, Intelligibility and
  Comfort, {\em IEEE Benelux Signal Processing Symposium}, Hilvarenbeek,
  Netherlands, April 2004
\item
  Rob de Vries and Bert de Vries, Toward SNR-Loss Restoration in Digital
  Hearing Aids, {\em ICASSP 2002}, Orlando, FL, May 2002
\item
  Bert de Vries, Jos Leenen, A Low Power Digital AGC Circuit for Dynamic
  Range Control of an A/D Converter, {\em International Hearing Aids
  Research (IHCON) Conference 2000}, Lake Tahoe (CA), August 2000
\item
  Lucas Parra, Clay Spence and Bert de Vries, Convolutive Blind Source
  Separation based on Multiple Decorrelation, {\em IEEE workshop on
  Neural Networks for Signal Processing VIII}, pp.23-32, Cambridge, UK,
  1998 {\bf {[}93{]}}
\item
  Bert de Vries, Blind Signal Processing for Hearing Aids, {\em NIH
  Hearing Aids Improvement Conference}, Bethesda, MA, May 1997
\item
  Bert de Vries, Adaptive Gamma Filters for Miniature Hearing Aids,
  {\em NIH Hearing Aids Improvement Conference}, Bethesda, MA, May 1997
\item
  Bert de Vries, Adaptive rank filtering based on error minimization,
  {\em ICASSP-97}, Munich, April 1997
\item
  Lucas Parra, Clay Spence, Bert De Vries, Convolutive Source Separation
  and Signal Modeling with Maximum Likelihood, {\em International
  Symposium on Intelligent Systems} (ISIS'97), Regio Calabria, Italy,
  1997
\item
  Q. Lin et al., Robust distant-talking speech recognition,
  {\em ICASSP-96}, Atlanta,GA, May 1996
\item
  Bert de Vries et al., Neural network speech enhancement for noise
  robust speech recognition, {\em International Workshop on Applications
  of Neural Networks to Telecommunications}, Sweden, May 1995
\item
  Lin et al., Experiments on distant-talking speech recognition,
  {\em ARPA Workshop on Spoken Language Technology}, Austin, TX, January
  1995
\item
  Qiguang Lin et al., System of microphone arrays and neural networks
  for robust speech recognition in multimedia environments, Proceedings
  {\em International Conference on Spoken Language Processing},
  Yokohama, Japan, September 1994
\item
  Bert de Vries, Gradient-based adaptation of network structure,
  {\em International Conference on Artificial Neural Networks 94},
  Sorrento, Italy, May 94
\item
  Che et al., Microphone Arrays and Neural Networks for Robust Speech
  Recognition, {\em ARPA Workshop on Human Language Technology},
  Princeton, NJ, March 1994
\item
  Bert de Vries et al., An application of Gamma delay lines to
  \quotation{BDG} phoneme classification, {\em Government Microcircuit
  Applications Conference proceedings}, New Orleans, LA, November 1993
\item
  Bert de Vries, Time-varying neural networks for large tasks,
  {\em International Conference on Artificial Neural Networks
  proceedings}, Amsterdam, the Netherlands, September 13-16, 1993
\item
  J.C. Principe et al., Backpropagation through time with fixed memory
  size requirements, {\em Proceedings of Workshop on Neural Networks for
  Signal Processing}, Linthicum Heights, MD, USA, Sep. 1993
\item
  Bert de Vries et al.,Learning with target trajectory constraints for
  sequence classification tasks, {\em ICASSP-93}, Minneapolis, MN, April
  1993
\item
  Bert de Vries et al.,Short Term Memory Structures for Dynamic Neural
  Networks, {\em Asilomar-92} Conference proceedings, Pacific Grove, CA,
  1992
\item
  T. Oliveira a Silva et al., Generalized feedforward filters with
  complex poles, {\em Proceedings of the 1992 IEEE workshop on Neural
  Networks for Signal Processing}, Copenhagen, Denmark, 1992
\item
  Jyh-Ming Kuo, Jose Principe and Bert de Vries, Prediction of chaotic
  time series using recurrent networks, {\em Proc. of the 1992 IEEE
  workshop on Neural Networks for Signal Processing}, 1992
\item
  Jose Principe, Bert de Vries and Pedro G. de Oliveira, Generalized
  feedforward structures: a new class of adaptive filters,
  {\em ICASSP-92}, San Francisco, vol.~IV, pp.~245-248, 1992
\item
  T. Oliveira e Silva, P. Guedes de Oliveira, J. C. Principe and B. de
  Vries, A Complex Pole Extension to the Gamma Filter, {\em The INESC
  Journal of Research and Development}, vol.~3, no. 1, pp.~35-41,
  Jan./Jun. 1992
\item
  Bert de Vries et al., Adaline with adaptive recursive memory,
  {\em Proceedings IEEE workshop on signal processing}, Princeton, NJ,
  1991
\item
  Principe et al., Modeling applications with the focused gamma net,
  {\em NIPS-4 proceedings}, Denver, CO, 1992
\item
  Bert de Vries et al., Some practical issues concerning the gamma
  neural net, {\em Proceedings IJCNN-91}, Seattle, WA, 1991
\item
  Bert de Vries and Jose Principe, A theory for neural nets with time
  delays, {\em NIPS-3 Proceedings}, Denver, 1991 {\bf {[}63{]}}
\item
  Bert de Vries et al., Neural net models for temporal processing,
  {\em Proceedings nineth southern biom. eng. conference}, Miami, FL,
  1991
\item
  Bert de Vries et al., A new neural net model for temporal processing,
  {\em 12th ann. int. conf. IEEE on the eng. in medicine and biology
  society}, Philadelphia, PA, 1990
\item
  Bert de Vries et al., Artificial neural networks as a computational
  paradigm for detection of anaesthetic complications, {\em Computers in
  Anesthesia 10}, New Orleans, LA, 1989
\item
  Bert de Vries et al., Distribution of anesthesia related occurrences
  during surgical operations, {\em Anesthesiology review} 14 (6), 1987
\stopitemize

\stoptext
